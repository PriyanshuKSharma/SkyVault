\documentclass[12pt,a4paper]{article}
\usepackage[margin=1in]{geometry}
\usepackage{graphicx}
\usepackage{hyperref}
\usepackage{titlesec}
\usepackage{enumitem}
\usepackage{fancyvrb} % For verbatim environments in boxes
\usepackage{tcolorbox} % For creating colored boxes
\usepackage{listings} % For code listings

\title{\textbf{SkyVault Project Report}}
\author{Priyanshu Kumar Sharma\\ BTech CTIS, Sem-5\\ Year-3   Sem-5\\  URN: 2022-B-17102004A\\ Information Security Applications\\ Course Code: MI300E\\ Prof Prachi Shukla}
\date{\today}

\begin{document}

% Add the image at the top of the document

% Now, create the title
\maketitle

\section*{1. Introduction}
SkyVault is a secure personal cloud storage solution that allows users to upload, manage, and access files privately. It prioritizes data security through encryption and authentication, giving users complete control over their data. Built with Flask and the \texttt{cryptography} library, SkyVault provides an accessible and secure alternative to third-party cloud services.

\section*{2. Objectives}
The main objectives of the project are:
\begin{itemize}
    \item Create a private cloud storage solution.
    \item Encrypt uploaded files for data confidentiality.
    \item Provide a user-friendly interface for file management.
    \item Implement robust error handling and security features.
\end{itemize}

\section*{3. Features}
SkyVault includes:
\begin{itemize}
    \item \textbf{User Authentication}: Secure login with hashed passwords.
    \item \textbf{File Encryption}: Files are encrypted using symmetric encryption.
    \item \textbf{File Management}: Upload, view, download, and delete files.
    \item \textbf{Error Handling}: Custom 404 and 500 error pages.
    \item \textbf{Secure Storage}: Encrypted files stored in a protected directory.
\end{itemize}

\section*{4. Project Structure}
The project repository is available at: \href{https://github.com/PriyanshuKSharma/SkyVault.git}{SkyVault GitHub Repository}. The directory structure is:

\begin{tcolorbox}[colback=blue!5!white, colframe=blue!75!black, sharp corners, boxrule=0.5mm]
\begin{Verbatim}
project/
├── app.py                   # Main application file
├── uploads/                 # Original uploaded files
├── encrypted_files/         # Encrypted files directory
├── templates/               # HTML templates
│   ├── index.html           # Homepage for file uploads
│   ├── files.html           # View uploaded files
│   ├── login.html           # Login page
│   ├── 404.html             # Custom 404 error page
│   ├── 500.html             # Custom 500 error page
└── static/                  # Static files (CSS, JS)
    └── css/
        └── styles.css       # Custom styles
\end{Verbatim}
\end{tcolorbox}

\section*{5. Implementation Details}
\begin{enumerate}[label=\arabic*.]
    \item \textbf{File Upload and Encryption}:
    Uploaded files are encrypted using the \texttt{cryptography} library before being stored in the \texttt{encrypted\_files/} directory.
    \item \textbf{User Authentication}:
    Passwords are securely hashed using \texttt{bcrypt}, and sessions are managed for logged-in users.
    \item \textbf{File Management}:
    Users can view, download, and delete their files through an intuitive interface.
    \item \textbf{Error Handling}:
    Custom error pages provide meaningful feedback for common issues.
\end{enumerate}

\section*{6. Usage}
\begin{tcolorbox}[colback=gray!5!white, colframe=gray!75!black, sharp corners, boxrule=0.5mm]
\begin{enumerate}
    \item Clone the repository:
    \begin{Verbatim}
    git clone https://github.com/PriyanshuKSharma/SkyVault.git
    cd SkyVault
    \end{Verbatim}
    \item Install dependencies:
    \begin{Verbatim}
    pip install -r requirements.txt
    \end{Verbatim}
    \item Run the application:
    \begin{Verbatim}
    python app.py
    \end{Verbatim}
    \item Access the application at \texttt{http://127.0.0.1:5000/}.
\end{enumerate}
\end{tcolorbox}

\section*{7. Future Enhancements}
\begin{itemize}
    \item Multi-user support for separate storage spaces.
    \item File sharing capabilities.
    \item Two-factor authentication (2FA).
    \item Mobile app development for increased accessibility.
\end{itemize}

\section*{8. Conclusion}
SkyVault offers a secure and private alternative to traditional cloud storage solutions. By combining encryption and user-friendly design, it ensures data confidentiality while providing an efficient file management system. For source code and further details, refer to the GitHub repository: \href{https://github.com/PriyanshuKSharma/SkyVault.git}{SkyVault}.

\section*{9. References}
\begin{itemize}
    \item Flask Documentation: \url{https://flask.palletsprojects.com}
    \item Cryptography Library: \url{https://cryptography.io}
    \item Bootstrap Framework: \url{https://getbootstrap.com}
\end{itemize}

\section*{10. Docker Integration}
SkyVault can be containerized using Docker for seamless deployment and scalability. Follow the steps below to run SkyVault using Docker.

\begin{tcolorbox}[colback=gray!5!white, colframe=gray!75!black, sharp corners, boxrule=0.5mm]
\begin{Verbatim}
# Step 1: Build the Docker image
docker build -t skyvault .

# Step 2: Run the Docker container
docker run -d -p 5000:5000 --name skyvault-container skyvault

# Step 3: Stop the container (if needed)
docker stop skyvault-container

# Step 4: Restart the container
docker start skyvault-container

# Step 5: Remove the container (if no longer needed)
docker rm skyvault-container
\end{Verbatim}
\end{tcolorbox}

\textbf{Dockerfile:}
Below is the Dockerfile used for creating the Docker image for SkyVault.

\begin{tcolorbox}[colback=blue!5!white, colframe=blue!75!black, sharp corners, boxrule=0.5mm]
\begin{Verbatim}
# Use an official Python runtime as a parent image
FROM python:3.10-slim

# Set the working directory in the container
WORKDIR /app

# Copy the current directory contents into the container at /app
COPY . /app

# Install any needed packages specified in requirements.txt
RUN pip install --no-cache-dir -r requirements.txt

# Make port 5000 available to the world outside this container
EXPOSE 5000

# Define environment variable
ENV FLASK_APP=app.py

# Run the application
CMD ["flask", "run", "--host=0.0.0.0"]
\end{Verbatim}
\end{tcolorbox}

\end{document}
``` 
